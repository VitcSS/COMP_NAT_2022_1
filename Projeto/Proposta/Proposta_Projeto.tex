%% ------------- Portuguese version ------------
\documentclass{sbrt}
\usepackage[english,brazil]{babel}
\usepackage[utf8]{inputenc}
\newtheorem{theorem}{Teorema}
%% ---------------------------------------------

%% If writing in English, remove the lines above
%% and uncomment the lines below

%% ------------- English version ---------------
%\documentclass[english]{sbrt}
%\usepackage[english]{babel}
%\usepackage[utf8]{inputenc}
%\newtheorem{theorem}{Theorem}
%% ---------------------------------------------

\begin{document}

\title{Redes de Busca de Trajetória para Modelos de Programação Genética }

\author{Clara Amorim Bacha de Almeida, Vítor Corrêa Silva
% \thanks{teste}%
}

\maketitle

% \markboth{teste}{}


%% If writing in English, remove both 'resumo' and 'chave'
%% ------------- Portuguese version ------------
\begin{resumo}
A proposta apresentada consiste na geração de grafos que representaram o funcionamento de um modelo de programação genética no espaço.
Visando permitir a representação gráfica dos resultados obtidos e tomar o projeto computacionalmente eficiente, será executada uma redução 
de dimensionalidade sobre as árvores o que trará estes dados para o espaço graficamente representável (até três dimensões).
\end{resumo}
\begin{chave}
Search Trajectory Networks, Redução de Dimensionalidade, Programação Génetica
\end{chave}
%% ---------------------------------------------


% \begin{abstract}
% This document is an example of how to use the \LaTeX\ style sbrt.cls to prepare a paper for submission to SBrT~2022. The abstract must have at most 100 words.

% For papers written in English, please drop both sections above (Resumo and Palavras-Chave).
% \end{abstract}
% \begin{keywords}
% Paper template, \LaTeX, SBrT~2022.
% \end{keywords}


\section{Introdução}

Quando se trabalha com com modelos de programação génetica, assim como em toda a área de computação natural, 
não é tangível como o modelo se comporta do ponto de vista prático, deixando de lado as metafóras e analogias 
e condiderando somente o processo computacional. Pensando isso Gabriela Ochoa, Katherine M. Malan e Christian 
Blum desenvolveram o que chamaram de "Search Trajectory Networks" que iremos traduzir como "Redes de Busca de 
Trajetória", essas redes são geradas átraves usando a saída da iteração como nó.

Até o momento as redes foram implementadas visando seu uso em algoritmos em que a sáida é essencialmente 
um valor númerico.  O objetivo deste trabalho será expandir esse conceito para modelos de programação 
genética aonde a saída não se resume a um valor númerico mas sim a um método.

\section{Metodologia}

Uma das formas mais comuns de se representar um modelo de programação génetica é se utilizando de árvores, essa é a forma que 
será a forma que será utilizada nesse trabalho. Árvores por natureza já podem ser implementadas graficamente em um espaço de 
até três dimensões, entretanto como a árvore representa o que será referido como "reposta atual" do algoritmo para a iteração que 
a gera. Para representar o comportamento de um dado algoritmo de programação genética será necessário de \(n\) árvores que representam
a resposta atual de cada uma das \(n\) iterações.

Sendo assim é necessário reduzir dimensionalmente a árvore para algo que possa ser representado como um ponto no espaço. Para isso será usado
um algoritmo de redução de dimensionalidade, como por exemplo PCA,para tal redução. Com as árvores convertidas para pontos é possível então gerar
um grafo direcionado que representará o comportamento do modelo. 


\begin{thebibliography}{99}
\bibitem{ref1} L. Lamport, \textit{A Document Preparation System: \LaTeX, User's
Guide and Reference Manual}. Addison Wesley Publishing Company,
1986.
% \bibitem{ref2} F. C. Silva e J. J. Sousa, ``Esta referência é apenas um exemplo," ~\textit{Revista de Exemplos}, v. 5, pp. 52--55, Maio
% 1999.
\end{thebibliography}


% \appendix
% Inserir as informações referentes aos apêndices aqui.


\end{document}